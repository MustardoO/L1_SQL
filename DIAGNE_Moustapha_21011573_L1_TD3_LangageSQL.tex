\documentclass[10pt,a4paper]{article}
\usepackage[utf8]{inputenc}
%\usepackage{setspace}	\singlespacing % Gère l'interlignage
\usepackage[english,french]{babel}
\usepackage[T1]{fontenc}
\usepackage[left=2cm,right=2cm,top=2cm,bottom=2cm]{geometry}
\usepackage{enumitem} %enumitem plus complet que enumerate
\usepackage{hyperref} %Pack pour les liens hypertexte
	\hypersetup{colorlinks=true,urlcolor=blue,linkcolor=blue}
\usepackage{tcolorbox} % permet la création de cadre colorés
\usepackage{graphicx}
\usepackage{float} % Pour pouvoir utiliser H

\title{TD3 Base de données : le langage SQL}	
\author{Diagne Moustapha id°21011573}
\date{\today}

\setlength\parindent{0pt} % \noindent pour tout le doc

\begin{document}
	
	%gainde@gainde-Inspiron-3505:~$ sudo -u postgres -i
	%[sudo] Mot de passe de gainde : 
	%postgres@gainde-Inspiron-3505:~$ psql
	%psql (12.9 (Ubuntu 12.9-0ubuntu0.20.04.1))
	%Type "help" for help.
	%postgres-# \c ma_bdd ---> \c permet de se connecter
	%You are now connected to database "ma_bdd" as user "postgres".
	
	\maketitle
	\pagebreak
	
	\section*{Exercice 1 : Création de tables}
	
	%\begin{tcolorbox}
	%Énoncé :
	%\tcblower
	1. Créer la table EMP dont la clé est NoEMP et les attributs ont pour types :
	
	%\end{tcolorbox}
	
	\begin{figure}[H]
		\centering
		\includegraphics[width=0.7\linewidth]{Captures/Exo1/Énoncé_table_exo1_question1}
		\label{fig:enoncetableexo1question1}
	\end{figure}


	\begin{figure}[H]
		\centering
		\includegraphics[width=0.7\linewidth]{Captures/Exo1/REPONSE_Creation_Table_exo1_question1}
		\label{fig:reponsecreationtableexo1question1}
	\end{figure}

	2. Remplir cette table par les données suivantes :
	
	\begin{figure}[H]
		\centering
		\includegraphics[width=0.7\linewidth]{Captures/Exo1/Énoncé_table_exo1_question2}
		\label{fig:enoncetableexo1question2}
	\end{figure}
	
	\begin{figure}[H]
		\centering
		\includegraphics[width=0.7\linewidth]{Captures/Exo1/REPONSE_Creation_Table_exo1_question2}
		\label{fig:reponsecreationtableexo1question2}
	\end{figure}
	
	3. Créer la table DEPT dont la clé est NoDEPT et les attributs ont pour types :
	
	
	\begin{figure}[H]
		\centering
		\includegraphics[width=0.4\linewidth]{Captures/Exo1/Énoncé_table_exo1_question3}
		\label{fig:enoncetableexo1question3}
	\end{figure}
	
	
	\begin{figure}[H]
		\centering
		\includegraphics[width=0.7\linewidth]{Captures/Exo1/REPONSE_Creation_Table_exo1_question3}
		\label{fig:reponsecreationtableexo1question3}
	\end{figure}
	
	4. Remplir cette table par les données suivantes :
	
	\begin{figure}[H]
		\centering
		\includegraphics[width=0.35\linewidth]{Captures/Exo1/Énoncé_table_exo1_question4}
		\label{fig:enoncetableexo1question4}
	\end{figure}
	
	\begin{figure}[H]
	\begin{minipage}{0.45\textwidth}
		\centering
		\includegraphics[width=1\linewidth]{Captures/Exo1/REPONSE_Creation_Table_exo1_question4-1}
		\label{fig:reponsecreationtableexo1question4-1}
	\end{minipage}
	\begin{minipage}{0.45\textwidth}
		\centering
		\includegraphics[width=0.9\linewidth]{Captures/Exo1/REPONSE_Creation_Table_exo1_question4-2}
		%\caption{}
		\label{fig:reponsecreationtableexo1question4-2}
	\end{minipage}
	\end{figure}

	5. Faire en sorte que la clé étrangère soit correctement définie, après avoir défini et rempli les deux tables.

	\begin{figure}[H]
		\centering
		\includegraphics[width=0.7\linewidth]{Captures/Exo1/REPONSE_Creation_Table_exo1_question5}
		%\caption{}
		\label{fig:reponsecreationtableexo1question5}
	\end{figure}
	
	\bigskip
	\section*{Exercice 2: Requêtes SQL simples}
	
	1. Afficher tous les tuples de la table EMP:
	
	\begin{figure}[H]
		\centering
		\includegraphics[width=0.7\linewidth]{Captures/Exo2/Rep1}
		%\caption{}
		\label{fig:rep1}
	\end{figure}
	
	2. À partir du contenu de la table EMP afficher les noms et les emplois de tous.
	
	\begin{figure}[H]
		\centering
		\includegraphics[width=0.4\linewidth]{Captures/Exo2/Rep2}
		%\caption{}
		\label{fig:rep2}
	\end{figure}
	
	3. Afficher les tuples correspondant au département 20 dans la table EMP.
		
	\begin{figure}[H]
		\centering
		\includegraphics[width=0.7\linewidth]{Captures/Exo2/Rep3}
		%\caption{}
		\label{fig:rep3}
	\end{figure}
	
	4. Rechercher et afficher les noms et salaires des employés exerçant en tant que vendeur et ayant une commission.
	
	\begin{figure}[H]
		\centering
		\includegraphics[width=0.55\linewidth]{Captures/Exo2/Rep4}
		%\caption{}
		\label{fig:rep4}
	\end{figure}
	
	5. Afficher tous les numéros de département. Est-il possible d’améliorer le contenu d’affichage ? Commenter.
	
	\begin{figure}[H]
		\centering
		\includegraphics[width=0.55\linewidth]{Captures/Exo2/Rep5}
		%\caption{}
		\label{fig:rep5}
	\end{figure}

	Il est effectivement possible d'améliorer l'affichage en utilisant GROUP BY (supprime les doublons) et ORDER BY (pour le tri).\\
	
	6. Lister les gestionnaires. On ne s’intéressera qu’à leurs noms, leurs salaires et leurs commissions.
		
	\begin{figure}[H]
		\centering
		\includegraphics[width=0.45\linewidth]{Captures/Exo2/Rep6}
		%\caption{}
		\label{fig:rep6}
	\end{figure}
	
	7. Déterminer la ville du département de ASMA.
	
	\begin{figure}[H]
		\centering
		\includegraphics[width=0.55\linewidth]{Captures/Exo2/Rep7}
		%\caption{}
		\label{fig:rep7}
	\end{figure}
	
	8. On s’intéresse aux employés gagnant entre 1000 e et 3000 e. Proposer une requête assurant cette tâche.
	
	\begin{figure}[H]
		\centering
		\includegraphics[width=0.7\linewidth]{Captures/Exo2/Rep8}
		%\caption{}
		\label{fig:rep8}
	\end{figure}
	
	9. Lister les noms des départements dont le nom de ville commence par R ou D.
		
	\begin{figure}[H]
		\centering
		\includegraphics[width=0.6\linewidth]{Captures/Exo2/Rep9}
		%\caption{}
		\label{fig:rep9}
	\end{figure}
		
	10. Lister les noms des départements dont le nom de ville contient un R ou un D.
	
	\begin{figure}[H]
		\centering
		\includegraphics[width=0.55\linewidth]{Captures/Exo2/Rep10}
		%\caption{}
		\label{fig:rep10}
	\end{figure}
	
	11. Dans l’ordre décroissant des salaires, lister les noms, les fonctions et les salaires des employés gagnant moins de
	3000 e.
		
	\begin{figure}[H]
		\centering
		\includegraphics[width=0.4\linewidth]{Captures/Exo2/Rep11}
		%\caption{}
		\label{fig:rep11}
	\end{figure}
	
	12. Dans l’ordre décroissant des numéros des départements ensuite dans l’ordre croissant des salaires, lister les noms,
	les fonctions, les salaires des employés et les numéros des départements.
	
	\begin{figure}[H]
		\centering
		\includegraphics[width=0.5\linewidth]{Captures/Exo2/Rep12}
		%\caption{}
		\label{fig:rep12}
	\end{figure}
		
	13. Dans l’ordre décroissant des noms des départements ensuite dans l’ordre croissant des salaires, lister les noms, les fonctions, les salaires des employés et les noms des départements.
	
	\begin{figure}[H]
		\centering
		\includegraphics[width=0.6\linewidth]{Captures/Exo2/Rep13}
		%\caption{}
		\label{fig:rep13}
	\end{figure}
	
	\bigskip
	\section*{Exercice 3: Création et mise à jour}
	
	1. On souhaite revaloriser le salaire de tous les employés de 200e chacun. Proposer l’instruction permettant de le faire sur la table EMP.
	
	\begin{figure}[H]
		\centering
		\includegraphics[width=0.7\linewidth]{Captures/Exo3/1}
		%\caption{}
		\label{fig:1}
	\end{figure}
	
	2. Insérer un nouveau tuple dans la table DEPT. Le contenu du tuple est laissé à votre choix.
	
	\begin{figure}[H]
		\centering
		\includegraphics[width=0.6\linewidth]{Captures/Exo3/2}
		%\caption{}
		\label{fig:2}
	\end{figure}
	
	3. Écrire et exécuter l’instruction de création de la table COMMISSION dont les attributs sont :
	
	\begin{itemize}
		\item[$(a)$] NomEmp de type chaîne de caractères (VARCHAR(10))
		\item[$(b)$] Emploi de type chaîne de caractères (VARCHAR(10))
		\item[$(c)$] Salaire de type réel (number(7,2))
		\item[$(d)$] ValComm de type réel (Number(4,2))
	\end{itemize}
	
	\begin{figure}[H]
		\centering
		\includegraphics[width=0.7\linewidth]{Captures/Exo3/3}
		%\caption{}
		\label{fig:3}
	\end{figure}

	4. Insérer plusieurs tuples dans cette table.
	
	\begin{figure}[H]
		\centering
		\includegraphics[width=0.7\linewidth]{Captures/Exo3/4}
		%\caption{}
		\label{fig:4}
	\end{figure}
	
	5. Supprimer de la table COMMISSION, le tuple correspondant à un nom d’employé (au choix).
	
	\begin{figure}[H]
		\centering
		\includegraphics[width=0.55\linewidth]{Captures/Exo3/5}
		%\caption{}
		\label{fig:5}
	\end{figure}
	
	6. Insérer dans cette table les données de la table EMP dont les employés ont une commission.
	
	\begin{figure}[H]
		\centering
		\includegraphics[width=0.6\linewidth]{Captures/Exo3/6}
		%\caption{}
		\label{fig:6}
	\end{figure}	
	
	7. Vider la table COMMISSION ensuite supprimer là.
	
	\begin{figure}[H]
		\centering
		\includegraphics[width=0.55\linewidth]{Captures/Exo3/7}
		%\caption{}
		\label{fig:7}
	\end{figure}
	
	8. Recréer la table COMMISSION en y chargeant directement les données de la table EMP dont les employés ont une
	commission.
	
	\begin{figure}[H]
		\centering
		\includegraphics[width=0.55\linewidth]{Captures/Exo3/8}
		%\caption{}
		\label{fig:8}
	\end{figure}
	
	\bigskip
	\section*{Exercice 4: Fonctions d’agrégat, ANY, ALL, GROUP BY, HAVING, etc.}
	
	1. Lister le nombre d’employés gagnant plus que le minimum de tous les salaires.
	
	\begin{figure}[H]
		\centering
		\includegraphics[width=0.55\linewidth]{Captures/Exo4/4_1}
		%\caption{}
		\label{fig:41}
	\end{figure}
		
	2. Afficher le nombre de départements différents dans la table EMP.
	
	\begin{figure}[H]
		\centering
		\includegraphics[width=0.55\linewidth]{Captures/Exo4/4_2}
%		\caption{}
		\label{fig:42}
	\end{figure}
	
	3. Afficher le salaire moyen dans la table EMP
	
	\begin{figure}[H]
		\centering
		\includegraphics[width=0.5\linewidth]{Captures/Exo4/4_3}
		\label{fig:43}
	\end{figure}
	
	4. Lister tout employé dont le salaire est supérieur à au moins un salaire du département 30.
	
	\begin{figure}[H]
		\centering
		\includegraphics[width=0.7\linewidth]{Captures/Exo4/4_4}
%		\caption{}
		\label{fig:44}
	\end{figure}
	
	5. Lister tout employé du département 10 dont le salaire est supérieur à tous ceux du département 20.
	
	\begin{figure}[H]
		\centering
		\includegraphics[width=0.7\linewidth]{Captures/Exo4/4_5}
%		\caption{}
		\label{fig:45}
	\end{figure}
	
	6. Lister les salaires minimal et maximal pour chaque département.
	
	\begin{figure}[H]
		\centering
		\includegraphics[width=0.55\linewidth]{Captures/Exo4/4_6}
%		\caption{}
		\label{fig:46}
	\end{figure}

	7. Lister le salaire moyen par emploi (EMPLOI).
	
	\begin{figure}[H]
		\centering
		\includegraphics[width=0.7\linewidth]{Captures/Exo4/4_7}
%		\caption{}
		\label{fig:47}
	\end{figure}
	
	8. Lister les salaires minimal et maximal pour chaque département ayant au moins 2 employés.
		
	\begin{figure}[H]
		\centering
		\includegraphics[width=0.55\linewidth]{Captures/Exo4/4_8}
%		\caption{}
		\label{fig:48}
	\end{figure}

	9. Lister les départements ayant le minimum des salaires supérieur à la moyenne des salaires des employés de bureau
	(FONCTIONNAIRE).
	
	\begin{figure}[H]
		\centering
		\includegraphics[width=0.7\linewidth]{Captures/Exo4/4_9}
%		\caption{}
		\label{fig:49}
	\end{figure}
	
	10. Donner les noms des employés par ordre alphabétique.
	
	\begin{figure}[H]
		\centering
		\includegraphics[width=0.35\linewidth]{Captures/Exo4/4_10}
%		\caption{}
		\label{fig:410}
	\end{figure}
	
	11. Donner les noms des employés par ordre alphabétique inversé.
	
	\begin{figure}[H]
		\centering
		\includegraphics[width=0.35\linewidth]{Captures/Exo4/4_11}
%		\caption{}
		\label{fig:411}
	\end{figure}
	
	12. Lister les noms des employés dont le nom de manager commence par un H ou un S ou un un M.
	
	\begin{figure}[H]
		\centering
		\includegraphics[width=0.5\linewidth]{Captures/Exo4/4_12}
%		\caption{}
		\label{fig:412}
	\end{figure}

	13. Lister les noms des employés dont le nom de manager se termine par un E ou un M ou un un A.
	
	\begin{figure}[H]
		\centering
		\includegraphics[width=0.5\linewidth]{Captures/Exo4/4_13}
%		\caption{}
		\label{fig:413}
	\end{figure}

	14. Donner les informations sur les employés par ordre décroissant de date d’embauche puis par ordre alphabétique de
	nom.
	
	\begin{figure}[H]
		\centering
		\includegraphics[width=0.7\linewidth]{Captures/Exo4/4_14}
%		\caption{}
		\label{fig:414}
	\end{figure}
	
	15. Présenter les employés par groupes de même valeur de salaire, et par ordre décroissant de ces valeurs.
		
	\begin{figure}[H]
		\centering
		\includegraphics[width=0.45\linewidth]{Captures/Exo4/4_15}
%		\caption{}
		\label{fig:415}
	\end{figure}
	
	
	16. Représenter les employés groupés selon les noms de département ordonnés dans l’ordre alphabétique inverse.
	
	\begin{figure}[H]
		\centering
		\includegraphics[width=0.7\linewidth]{Captures/Exo4/4_16}
%		\caption{}
		\label{fig:416}
	\end{figure}
	
	
	17. Représenter les employés groupés selon le mois et l’année de leur date d’embauche (ordre chronologique).
	
	\begin{figure}[H]
		\centering
		\includegraphics[width=0.7\linewidth]{Captures/Exo4/4_17}
%		\caption{}
		\label{fig:417}
	\end{figure}
	
	18. Afficher le salaire moyen par job.
	
	\begin{figure}[H]
		\centering
		\includegraphics[width=0.55\linewidth]{Captures/Exo4/4_18}
%		\caption{}
		\label{fig:418}
	\end{figure}
	
	19. Afficher les noms des employés groupés par job puis par nom de département.
	
%	\begin{figure}[H]
%		\centering
%		\includegraphics[width=0.7\linewidth]{Captures/Exo4/4_19}
%		\caption{}
%		\label{fig:419}
%	\end{figure}

	\begin{figure}[H]
		\centering
		\includegraphics[width=0.6\linewidth]{Captures/Exo4/4_19bis}
%		\caption{}
		\label{fig:419bis}
	\end{figure}

	20. Afficher les nombres d’employés dirigés directement par chaque manager (ici on comprend manager comme un employé ayant son numéro dans la colonne mgr).
	
	\begin{figure}[H]
		\centering
		\includegraphics[width=0.7\linewidth]{Captures/Exo4/4_20}
%		\caption{}
		\label{fig:420}
	\end{figure}

	21. Afficher la ville et le nom des départements qui ont plus de quatre employés.
	
%	\begin{figure}[H]
%		\centering
%		\includegraphics[width=0.55\linewidth]{Captures/Exo4/4_21}
%		\caption{}
%		\label{fig:421}
%	\end{figure}	
	
	\begin{figure}[H]
		\centering
		\includegraphics[width=0.55\linewidth]{Captures/Exo4/4_21bis}
%		\caption{}
		\label{fig:421bis}
	\end{figure}
	
	22. Afficher le nom des départements où les employés gagnent en moyenne plus de 2500e.
	
	\begin{figure}[H]
		\centering
		\includegraphics[width=0.4\linewidth]{Captures/Exo4/4_22}
%		\caption{}
		\label{fig:422}
	\end{figure}
	
	23. Afficher le nombre d’employés par département et par profession.
	
	\begin{figure}[H]
		\centering
		\includegraphics[width=0.7\linewidth]{Captures/Exo4/4_23}
%		\caption{}
		\label{fig:423}
	\end{figure}
	
	24. Donner pour chaque département, son nom, le salaire minimum de ses employés, le salaire maximum, le salaire
	moyen, le nombre d’employés, le nombre d’employés touchant une commission (éventuellement nulle) et le nombre
	de jobs différents exercés dans ce département.
	
	\begin{figure}[H]
		\centering
		\includegraphics[width=0.9\linewidth]{Captures/Exo4/4_24}
%		\caption{}
		\label{fig:424}
	\end{figure}
	
	\bigskip
	\section*{Exercice 5: Autres aspects de SQL}	
	
	1. Créer une vue EMPDEPT qui contient la jointure naturelle des deux tables EMP et DEPT.
	
	\begin{figure}[H]
		\centering
		\includegraphics[width=0.9\linewidth]{Captures/Exo5/5_1}
%		\caption{}
		\label{fig:51}
	\end{figure}
	
	\begin{figure}[H]
		\centering
		\includegraphics[width=0.9\linewidth]{Captures/Exo5/5_1bis}
%		\caption{}
		\label{fig:51bis}
	\end{figure}
	
	2. En utilisant cette vue, afficher les noms d’employés touchant plus que 2000 e avec leurs villes.
	
	\begin{figure}[H]
		\centering
		\includegraphics[width=0.45\linewidth]{Captures/Exo5/5_2}
%		\caption{}
		\label{fig:52}
	\end{figure}
	
	3. Vider la table EMP.
	
	\begin{figure}[H]
		\centering
		\includegraphics[width=0.7\linewidth]{Captures/Exo5/5_3}
%		\caption{}
		\label{fig:53}
	\end{figure}
	
	4. Supprimer la table EMP.
		
	\begin{figure}[H]
		\centering
		\includegraphics[width=0.7\linewidth]{Captures/Exo5/5_4}
%		\caption{}
		\label{fig:54}
	\end{figure}
	
	5. Créer à nouveau la table EMP avec les mêmes attributs et les mêmes contraintes d’intégrité.
	
	\begin{figure}[H]
		\centering
		\includegraphics[width=0.9\linewidth]{Captures/Exo5/5_5}
%		\caption{}
		\label{fig:55}
	\end{figure}
	
	6. Supprimer la colonne COMM.
	
	\begin{figure}[H]
		\centering
		\includegraphics[width=0.7\linewidth]{Captures/Exo5/5_6}
%		\caption{}
		\label{fig:56}
	\end{figure}
	
	7. Renommer la colonne COMM en BONUS.
	
	\begin{figure}[H]
		\centering
		\includegraphics[width=0.7\linewidth]{Captures/Exo5/5_7}
%		\caption{}
		\label{fig:57}
	\end{figure}
	
	
	8. Ajouter une colonne à la table DEPT.
	
	\begin{figure}[H]
		\centering
		\includegraphics[width=0.9\linewidth]{Captures/Exo5/5_8}
%		\caption{}
		\label{fig:58}
	\end{figure}
	
	
	
\end{document}